\begin{center}
    \begin{longtable}[l]{| p{0.18\textwidth} | p{0.18\textwidth} | p{0.18\textwidth}*{12}{| p{0.01\textwidth} }| }
    \captionsetup{singlelinecheck=false, justification=raggedright, labelfont=bf}
    \caption{Schedule of Activities.}  \\
    \hline
    \textbf{\small OBJECTIVES} & \textbf{\small TARGET ACTIVITIES} & \textbf{\small TARGET ACCOMPLISHMENTS} {\footnotesize(quantify, if possible)} & \multicolumn{12}{|c|}{\textbf{\small YEAR 1}} \\
    \hline
    % The second line, with its five years of four quarters
      & & & {\scriptsize 1} & {\scriptsize 2} & {\scriptsize 3} & {\scriptsize 4} & {\scriptsize 5} & {\scriptsize 6} & {\scriptsize 7} & {\scriptsize 8} & {\scriptsize 9} & {\scriptsize 10} & {\scriptsize 11} & {\scriptsize 12} \\
    \hline
    % using the on macro to fill in twenty cells as `on'
    {\footnotesize Objective 1} & 
    {\scriptsize To develop an alternative approach to food classification using only multiple color spaces extracted from a food image.}
    \newline &
    {\scriptsize 1. Implement an initial prototype that test the feasibility of the study using Python, Keras, and TensorFlow.} \newline
    {\scriptsize 2. Using EfficientNetB0 in building the prototype to see the performance difference between the previous studies and the current study.} \newline
    {\scriptsize 3. Using the built prototype to test the feasibility of multi-color space input performance difference between the previous studies and the current study.}
        \off[9] \on[3]\\ 
    \hline
    {\footnotesize Objective 2} & 
    {\scriptsize To conduct an ablation study on the best 2-color space input combination for the Siamese-CNN model.}
    \newline &
    {\scriptsize Given the related literature on the multi color spaces, test and validate the results and perform ablation study on the 2 best color space combination to be utilized in the study.}
        \off[9] \on[3]\\ 
    \hline
    \end{longtable}
    \newpage
    \begin{longtable}{| p{0.18\textwidth} | p{0.18\textwidth} | p{0.18\textwidth}*{12}{|p{0.01\textwidth} }| }
    \hline
    \textbf{\small OBJECTIVES} & \textbf{\small TARGET ACTIVITIES} & \textbf{\small TARGET ACCOMPLISHMENTS} {\footnotesize(quantify, if possible)} & \multicolumn{12}{|c|}{\textbf{\small YEAR 2}} \\
    \hline
    % The second line, with its five years of four quarters
      & & & {\scriptsize 1} & {\scriptsize 2} & {\scriptsize 3} & {\scriptsize 4} & {\scriptsize 5} & {\scriptsize 6} & {\scriptsize 7} & {\scriptsize 8} & {\scriptsize 9} & {\scriptsize 10} & {\scriptsize 11} & {\scriptsize 12} \\
    \hline
    % using the on macro to fill in twenty cells as `on'
    {\footnotesize Objective 3} & 
    {\scriptsize To compare the outcome of the study to the currently accepted best models in the food classification problem. } 
    \newline & 
    {\scriptsize 1. Develop the true version of the model.} \newline
    {\scriptsize 2. Perform experiments based on the methodology.} \newline
    {\scriptsize 3. Conduct a comparison and analysis on the results taken from the experiment and the previous study.} \newline
    {\scriptsize 4. Validate the feasibility of the prototype that it can outperform current best models in food classification.}
        \on[3] \off[9]\\ 
    \hline
    
    {\footnotesize Objective 4} & 
    {\scriptsize To implement a Siamese-CNN deep learning model on food classification using a web-based application.}
        \newline & 
    {\scriptsize 1. Develop the front-end of the web-based system using the HTML, CSS, and JavaScript Stack.}\newline
    {\scriptsize 2. Develop the back-end using Flask.}\newline
    {\scriptsize 3. Deploy the web-based food classification system to production.} 
        \on[4] \off[8]\\ 
    \hline
    
    \end{longtable}
\end{center}