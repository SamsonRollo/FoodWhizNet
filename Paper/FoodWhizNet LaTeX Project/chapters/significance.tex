The significance of this study will be associated with the importance of food classification in our everyday life. May it be for professional use in the medical and research field or just for personal use of personal diet plans. Studies that mostly use food classification systems are those that concern the diet of patients with allergies or diabetes for instance. Despite its popularity in the latter fields, the solutions to the food classification problem do not always settle on surface-level use but also for conducting state-of-the-art studies that contribute to the body of knowledge. This is because of the problem's intrinsic high and low intra-class variability characteristics \cite{ciocca-2019}, as well as non-linearity \cite{islam-2018}. Therefore, it is the best avenue for conducting the feasibility of novel and hybrid solutions, especially on a multiple variation single input like the multi-color space input. Since the study explores an alternative way of solving this kind of problem, the choice of using the multi-color space as input is considered. This is to validate the claim of Gowda \& Yuan \cite{gowda-2019} on the effectiveness of using multiple color spaces in the image classification problem, only in our case, applied to the food classification problem. Addressing this gap can spurt future studies on the sole use of multi-color space in image classification problems. In addition to the use of multiple color spaces, the use of Siamese CNN as a model for this study justifies the use of the input. Siamese CNN is known for its robustness in checking for similarity and dissimilarity of two input images, modifying it to learn specific similarities between the color spaces of an image can produce a state-of-the-art study that can also revolutionize deep learning using novel and hybrid techniques. The outcome of this study can be used to validate existing studies as well as create a new node for food image classification on different feature inputs. The deployment of the web-based application will be also utilized in fields not limited to the restaurant industry, medicine, food technology, etc.