Food is essential to human survival as per Maslow’s Hierarchy of Needs \cite{Mcleod2023}. It sits at the very bottom of the hierarchy identified as a physiological need together with shelter, water, and clothing. Its heavy significance to humans creates a series of research that revolves mostly around its use, identification, and classification. One of the main research fields that can be related to food is the food classification problem. This is a sub-variant of the image classification problem where machines are taught to classify a certain food to either of the list of classes. This has a practical and beneficial application in various fields not limited to medicine, research, and the restaurant industry. For instance, the use of an image classification system in diet planning can be seen in \cite{brintha-2022,chun-2022,dong2019diet,de-kervenoael-2023}. 

In the food review with the application of deep learning of Zhou et al. \cite{zhou-2019}, most of the models that were developed focus on the conventional features that are present in the food images. Most models that were also developed for food classification systems are either using the naive single feature input CNN, or multiple feature input CNN. Features like color, shape, volume, composition, and texture. One of the distinguished studies was that of Al-Sarayreh et al. \cite{al-sarayreh-2018} which utilized both spectral and spatial features for its meat quality analysis using two unbalanced sub-networks of CNN. The same goes with Pandey et al. \cite{pandey-2017} and Martinel et al. \cite{martinel-2018} which can be generalized to a multilayered CNN pipeline. Using these premises we can arrive at classifying our research problem through our gap. That is the absence of the use of multiple variations of a single input using a balanced sub-network CNN, specifically the multiple color space input. We consider the use of the color feature as an input since it is the most distinguished among the features and multiple studies on multiple color space input for deep learning were also present.

In line with conducting a state-of-the-art alternative approach to classifying food images, the use of the Siamese CNN network model is considered to address the borderline capability of training, testing, and deploying the proposed web-based application. Utilizing a supervised deep-learning network such as CNN is best for this study as this is a sub-variant of the image classification problem \cite{sarker-2021}. To address the multiple variations from a single input, this study will be utilizing Siamese CNN, a specialized type of neural network that best works in similarity and dissimilarity problems that takes two inputs fed on two balanced sub-networks.

The proposed study aims to reach at least the existing best results which uses conventional CNN with inputs used to maximize the utilization of the previously mentioned features. 